\section{Appendix}
\label{sec:appendix}

%%%%%%%%%%%%%%%%%%%%%%%%%%%%%%%%%%%
\subsection{Developers Parameters}%
%%%%%%%%%%%%%%%%%%%%%%%%%%%%%%%%%%%
\begin {description}

\item {\texttt{-{}-askozia-confpage=<string>}} \newline
Name of the PHP page for downloading/uploading the XML Configuration file
\newline Default: \texttt{system\_backup.php}
\newline Example: \texttt{-{}-askozia-confpage=system\_backup.php}

\item {\texttt{-{}-askozia-realm=<string>}} \newline
Name of the authentication-realm of the AskoziaPBX.
\newline Default: \texttt{Web Server Authentication}
\newline Example: \texttt{-{}-askozia-realm='Web Server Authentication'}

\item {\texttt{-{}-sipp-path=<string>}} \newline
This parameter allows one to override the sipp executable which is used to execute the performance tests.
It is recommended to use the supplied version because it provides a standardized test base.
\newline Default: \texttt{./PERF\_TESTS/sipp} or, if not existing, the result of \texttt{which sipp}
\newline E.g. \texttt{-{}-sipp-path=../sipp} or \texttt{-{}-sipp-path=/tmp/sipp}

\item {\texttt{-{}-two-party-user-file=<string>}} \newline
For executing two-party tests with sipp, there has to be a so-called injection file.
This is a csv file which is used by sipp for generating multiple calls automaticly.
It is created by the script and contains all needed information (and not more)
for sipp. It is possible to change the filepath and -name of this file by specifying
this parameter and may be absolute or relative.
\newline Default: \texttt{./results/<testname>/Users\_two-party.csv}
\newline E.g. \texttt{-{}-users-2way-file=../Users\_two-party.csv}
\newline or \texttt{-{}-users-2way-file=/tmp/Users\_two-party.csv}

\item {\texttt{-{}-participants-user-file=<string>}} \newline
Please have a look at the parameter \texttt{-{}-users-2way-file}.
This parameter is the same only for conference tests with fixed number of rooms.
\newline Default: \texttt{./results/<testname>/Users\_conf\_room.csv}
\newline E.g. \texttt{-{}-users-conf-room-file=../Users\_conf\_room.csv}
\newline or \texttt{-{}-users-conf-room-file=/tmp/Users\_conf\_room.csv}

\item {\texttt{-{}-room-user-file=<string>}} \newline
Please have a look at the parameter \texttt{-{}-users-2way-file}.
This parameter is the same only for conference tests with fixed number of calls.
\newline Default: \texttt{./results/<testname>/Users\_conf\_call.csv}
\newline E.g. \texttt{-{}-users-conf-call-file=../Users\_conf\_call.csv}
\newline or \texttt{-{}-users-conf-call-file=/tmp/Users\_conf\_call.csv}

\item {\texttt{-{}-reg-scen=<string>}} \newline
This parameter specifies the path to the register scenario used by sipp.
The register scenario is needed for two-party tests only.
For more information, please have a look at chapter \ref{sec:two-party-test}.
\newline Default: \texttt{./PERF\_TEST\_FILES/Register.xml}
\newline E.g. \texttt{-{}-reg-scen=../Register.xml} or \texttt{-{}-reg-scen=/tmp/Register.xml}

\item {\texttt{-{}-dereg-scen=<string>}} \newline
This parameter specifies the path to the deregister scenario used by sipp.
The deregister scenario is needed for two-party tests only.
For more information, please have a look at chapter \ref{sec:two-party-test}.
\newline Default: \texttt{./PERF\_TEST\_FILES/Deregister.xml}
\newline E.g. \texttt{-{}-dereg-scen=../Deregister.xml} or \texttt{-{}-dereg-scen=/tmp/Deregister.xml}

\item {\texttt{-{}-acc-scen=<string>}} \newline
This parameter specifies the path to the accept scenario used by sipp.
The accept scenario is needed for two-party tests only.
For more information, please have a look at chapter \ref{sec:two-party-test}.
\newline Default: \texttt{./PERF\_TEST\_FILES/Accept.xml}
\newline E.g. \texttt{-{}-acc-scen=../Accept.xml} or \texttt{-{}-acc-scen=/tmp/Accept.xml}

\item {\texttt{-{}-inv-scen=<string>}} \newline
This parameter specifies the path to the invite scneario used by sipp.
The invite scenario is needed for every test type.
For more information, please have a look at the description of the
different testtypes (chapters \ref{sec:two-party-test}, \ref{sec:conf-rooms-test} and \ref{sec:conf-participants-test}).
\newline Default: \texttt{./PERF\_TEST\_Files/Invite.xml}
\newline E.g. \texttt{-{}-inv-scen=../Invite.xml} or \texttt{-{}-inv-scen=/tmp/Invite.xml}

\item {\texttt{-{}-sip-src-port=<number>}} \newline
Port of the local computer (testcomputer) to communicate with Askozia.
It is used for User A in two-party tests and for all users in conference tests.
During the tests, there was a softphone running in background for communication
in the office. Because of this, sipp was not able to reserve the usual sip port 5060.
\newline Default: \texttt{5061}
\newline E.g. \texttt{-{}-sip-src-port=5061}

\item {\texttt{-{}-sip-dst-port=<number>}} \newline
Port of the local computer (testcomputer) to communicate with Askozia,
but this time only for User B in two-party tests. The first sipp process
(User A) blocks one port for communication with AskoziaPBX,
so the second sipp process (User B) needs another one to talk to Askozia.
This is necessary for two-party tests only.
\newline Default: \texttt{5062}
\newline E.g. \texttt{-{}-sip-dst-port=5062}

\item {\texttt{-{}-rtp-src-port=<number>}} \newline
Port of the local computer for establishing RTP streams between the local testcomputer
and AskoziaPBX. This one is used by User A of two-party tests and by all users of
conference tests. Sipp was not able to use the standard port because of a
softphone running on the testcomputer in background.
\newline Default: \texttt{6020}
\newline E.g. \texttt{-{}-rtp-src-port=6020}

\item {\texttt{-{}-rtp-dst-port=<number>}} \newline
Port of the local computer for establishing RTP streams between the local testcomputer
and AskoziaPBX. This one is used by User B of two-party tests only, so it not needed
for conference testing.
\newline Default: \texttt{6030}
\newline E.g. \texttt{-{}-rtp-dst-port=6030}

\item {\texttt{-{}-restore=<string>}} \newline
After the test, the AskoziaPBX is strongly reconfigured. There are possibly hundreds
of testusers and some new conference rooms. To avoid cleaning up by hand, this parameter
helps to reconfigure the box after the test. There are three possible values:
\begin{description}
	\item [none] The AskoziaPBX is not reconfigured.
	\item [old-config] The AskoziaPBX is restored with the config existing before the test.
	\item [factory-defaults] The AskoziaPBX is set to factory-defaults.
\end{description}
Default: \texttt{old-config}
\newline E.g. \texttt{-{}-restore=none} or \texttt{-{}-restore=factory-defaults}

\item {\texttt{-{}-gnuplot-exe=<string>}} \newline
The path to the gnuplot executable for drawing graphs of the results at the end of the test.
Has to be installed (\texttt{which gnuplot} determines path) or specified if the graphs
should be drawed. If not installed and specified (undefined), there is no possibility to
draw the graphs. The test can nevertheless be executed.
\newline Default: result of \texttt{which gnuplot} or, if not existing, undefined
\newline E.g. \texttt{-{}-gnuplot-exe=../gnuplot} or \texttt{-{}-gnuplot-exe=/tmp/gnuplot}

\item {\texttt{-{}-two-party-user-file=<string>}} \newline
For executing two-party tests with sipp, there has to be a so-called injection file.
This is a csv file which is used by sipp for generating multiple calls automaticly.
It is created by the script and contains all needed information (and not more)
for sipp. It is possible to change the filepath and -name of this file by specifying
this parameter and may be absolute or relative.
\newline Default: \texttt{./results/<testname>/Users\_two-party.csv}
\newline E.g. \texttt{-{}-users-2way-file=../Users\_two-party.csv}
\newline or \texttt{-{}-users-2way-file=/tmp/Users\_two-party.csv}

\item {\texttt{-{}-participants-user-file=<string>}} \newline
Please have a look at the parameter \texttt{-{}-users-2way-file}.
This parameter is the same only for conference tests with fixed number of rooms.
\newline Default: \texttt{./results/<testname>/Users\_conf\_room.csv}
\newline E.g. \texttt{-{}-users-conf-room-file=../Users\_conf\_room.csv}
\newline or \texttt{-{}-users-conf-room-file=/tmp/Users\_conf\_room.csv}

\item {\texttt{-{}-room-user-file=<string>}} \newline
Please have a look at the parameter \texttt{-{}-users-2way-file}.
This parameter is the same only for conference tests with fixed number of calls.
\newline Default: \texttt{./results/<testname>/Users\_conf\_call.csv}
\newline E.g. \texttt{-{}-users-conf-call-file=../Users\_conf\_call.csv}
\newline or \texttt{-{}-users-conf-call-file=/tmp/Users\_conf\_call.csv}

\item {\texttt{-{}-reg-scen=<string>}} \newline
This parameter specifies the path to the register scenario used by sipp.
The register scenario is needed for two-party tests only.
For more information, please have a look at chapter \ref{sec:two-party-test}.
\newline Default: \texttt{./PERF\_TEST\_FILES/Register.xml}
\newline E.g. \texttt{-{}-reg-scen=../Register.xml} or \texttt{-{}-reg-scen=/tmp/Register.xml}

\item {\texttt{-{}-dereg-scen=<string>}} \newline
This parameter specifies the path to the deregister scenario used by sipp.
The deregister scenario is needed for two-party tests only.
For more information, please have a look at chapter \ref{sec:two-party-test}.
\newline Default: \texttt{./PERF\_TEST\_FILES/Deregister.xml}
\newline E.g. \texttt{-{}-dereg-scen=../Deregister.xml} or \texttt{-{}-dereg-scen=/tmp/Deregister.xml}

\item {\texttt{-{}-acc-scen=<string>}} \newline
This parameter specifies the path to the accept scenario used by sipp.
The accept scenario is needed for two-party tests only.
For more information, please have a look at chapter \ref{sec:two-party-test}.
\newline Default: \texttt{./PERF\_TEST\_FILES/Accept.xml}
\newline E.g. \texttt{-{}-acc-scen=../Accept.xml} or \texttt{-{}-acc-scen=/tmp/Accept.xml}

\item {\texttt{-{}-inv-scen=<string>}} \newline
This parameter specifies the path to the invite scneario used by sipp.
The invite scenario is needed for every test type.
For more information, please have a look at the description of the
different testtypes (chapters \ref{sec:two-party-test}, \ref{sec:conf-rooms-test} and \ref{sec:conf-participants-test}).
\newline Default: \texttt{./PERF\_TEST\_Files/Invite.xml}
\newline E.g. \texttt{-{}-inv-scen=../Invite.xml} or \texttt{-{}-inv-scen=/tmp/Invite.xml}

\item {\texttt{-{}-sip-src-port=<number>}} \newline
Port of the local computer (testcomputer) to communicate with Askozia.
It is used for User A in two-party tests and for all users in conference tests.
During the tests, there was a softphone running in background for communication
in the office. Because of this, sipp was not able to reserve the usual sip port 5060.
\newline Default: \texttt{5061}
\newline E.g. \texttt{-{}-sip-src-port=5061}

\item {\texttt{-{}-sip-dst-port=<number>}} \newline
Port of the local computer (testcomputer) to communicate with Askozia,
but this time only for User B in two-party tests. The first sipp process
(User A) blocks one port for communication with AskoziaPBX,
so the second sipp process (User B) needs another one to talk to Askozia.
This is necessary for two-party tests only.
\newline Default: \texttt{5062}
\newline E.g. \texttt{-{}-sip-dst-port=5062}

\item {\texttt{-{}-rtp-src-port=<number>}} \newline
Port of the local computer for establishing RTP streams between the local testcomputer
and AskoziaPBX. This one is used by User A of two-party tests and by all users of
conference tests. Sipp was not able to use the standard port because of a
softphone running on the testcomputer in background.
\newline Default: \texttt{6020}
\newline E.g. \texttt{-{}-rtp-src-port=6020}

\item {\texttt{-{}-rtp-dst-port=<number>}} \newline
Port of the local computer for establishing RTP streams between the local testcomputer
and AskoziaPBX. This one is used by User B of two-party tests only, so it not needed
for conference testing.
\newline Default: \texttt{6030}
\newline E.g. \texttt{-{}-rtp-dst-port=6030}

\item {\texttt{-{}-restore=<string>}} \newline
After the test, the AskoziaPBX is strongly reconfigured. There are possibly hundreds
of testusers and some new conference rooms. To avoid cleaning up by hand, this parameter
helps to reconfigure the box after the test. There are three possible values:
	\begin{description}
	\item [none] The AskoziaPBX is not reconfigured.
	\item [old-config] The AskoziaPBX is restored with the config existing before the test.
	\item [factory-defaults] The AskoziaPBX is set to factory-defaults.
	\end{description}
Default: \texttt{old-config}
\newline E.g. \texttt{-{}-restore=none} or \texttt{-{}-restore=factory-defaults}
\end{description}

\subsection{XML Scenarios}
\subsubsection{REGISTER Scenario}
\subsubsection{De-REGISTER Scenario}
\subsubsection{INVITE Scenario}
\subsubsection{ACCEPT Scenario}

\subsection{Example Configuration File}

\subsection{SIPP Manpage}

