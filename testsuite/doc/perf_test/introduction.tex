\section{Introduction}
\label{sec:introduction}

This subproject of AskoziaPBX was developed for executing performance and stress tests on different Askozia installations.
It was written by Mark Stephan \newline (mark.stephan@askozia.com) during a job as a student assistant in Spring/Summer 2010. 

% TODO
%It is very important to understand the differences between the three test-types (two-way-calls, conference calls with fixed number of
%calls and conference calls with fixed number of rooms). It is explained detailed in the first paragraphs of chapters
%\ref{sec:two-way}, \ref{sec:conf-call} and \ref{sec:conf-room}.

%%%%%%%%%%%%%%%%%%%%%%
\subsection{Problem}%%
%%%%%%%%%%%%%%%%%%%%%%
The AskoziaPBX software can be downloaded as a firmware image for embedded systems and as a live cd.
The live cd can be run on every normal computer, so the underlying hardware may have very different
performance (e.g., the same software can handle three, 30 or 300 parallel two-way-calls,
depending on the computer performance). For this reasion, we had to develop an algorithm to find out
how capable the current Askozia box is.
 
%%%%%%%%%%%%%%%%%%%%%%%
\subsection{Features}%%
%%%%%%%%%%%%%%%%%%%%%%%
The current testsuite supports the following features:

\begin{itemize}
\item completely automated testing of one AskoziaPBX
\item automatic configuration of the AskoziaPBX installations with the needed settings
\item three different types of tests:
	\begin{description}
	\item [two-participants tests] %TODO The testsuite establishes a variable number of A-to-B (or end-to-end) calls between the AskoziaPBX and the testsystem. This test simulates normal telephone calls between two persons.
	
	\item [conference rooms tests] %TODO The testsuite calls a variable number of different conference rooms with a fixed number of users in each room.

	\item [conference participants tests] %TODO The testsuite calls a fix number of conference rooms with a variable number of users in each rooms.
	\end{description}
	
\item monitoring of the CPU load of the AskoziaPBX caused by the testcalls
\item downloading the recorded CPU load data
\item interpretation and creation of graphs of the testresults 
\end{itemize}

\newpage
%%%%%%%%%%%%%%%%%%%%
\subsection{Usage}%%
%%%%%%%%%%%%%%%%%%%%
The script can be called from the command line as described below: 

\begin{lstlisting}[breaklines=true,label=code:script-usage,caption={Script usage} ]
./PERF_TEST <options>
perl PERF_TEST <options>

./PERF_TEST --local-ip=192.168.0.2
    --askozia-ip=192.168.0.1
    --max-two-party-test=30
    (two-participants test with maximal 30 users)

./PERF_TEST --local-ip=192.168.100.20
    --askozia-ip=192.168.100.200
    --max-conference-rooms-test=15
    (conference rooms test with maximal 15 rooms)

./PERF_TEST --local-ip=10.10.10.10
    --askozia-ip=10.10.10.5
    --max-conference-participants-test=40
    (conference participants test with maximal 40 users)

./PERF_TEST --local-ip=192.168.2.100
    --askozia-ip=192.168.2.1
    --max-two-party-test=30
    --max-conference-rooms-test=15
    --max-conference-participants-test=40
    (executes all three different tests sequentially)
\end{lstlisting}

The script's parameters can be classified in three sections: ``Necessary'', ``Optional'' and ``Developers''.
The first two groups are described below, the ``Developers'' parameters are listed in the appendix.
You have to be root to execute this script because \texttt{sipp} reserves port for its connection to Askozia.

%%%%%%%%%%%%%%%%%%%%%%%%%%%%%%%%%%%%%
\subsubsection{Necessary Parameters}%
%%%%%%%%%%%%%%%%%%%%%%%%%%%%%%%%%%%%%

\begin{description}
\item {\texttt{-{}-local-ip=<IP>}} \newline
The IP-adress of the testcomputer that executes the testscript.
<IP> stands for the address of the network interface connected to the AskoziaPBX.
\newline Default: undefined
\newline Example: \texttt{-{}-local-ip=192.168.0.2}
\end{description}

At least one of these three following parameters is necessary, too:
\begin{description}

%TODO 
\item {\texttt{-{}-max-two-party-test=<number>}}
\newline Default: undefined (no two-way tests)
\newline Example: \texttt{-{}-max-two-party-test=30}

\item {-{}-max-conference-rooms-test}
\newline Default: undefined (no conference rooms test)
\newline Example: \texttt{-{}-max-conference-rooms-test=15}

\item {-{}-max-conference-participants-test}
\newline Default: undefined (no conference participants tests)
\newline Example: \texttt{-{}-max-conference-participants-test=40}

\end{description}
%%%%%%%%%%%%%%%%%%%%%%%%%%%%%%%%%%%%
\subsubsection{Optional Parameters}%
%%%%%%%%%%%%%%%%%%%%%%%%%%%%%%%%%%%%
\begin{description}

\item {\texttt{-{}-askozia-ip=<IP>}} \newline
The IP-address of the AskoziaPBX installation that is to be tested.
\newline Default: \texttt{10.10.10.1}
\newline Example: \texttt{-{}-askozia-ip=192.168.0.1}

\item {\texttt{-{}-askozia-port=<number>}} \newline
The number of the webport of the AskoziaPBX.
\newline Default: \texttt{80}
\newline Example: \texttt{-{}-askozia-port=80}

\item {\texttt{-{}-askozia-user=<string>}} \newline
Name of the administrator user of the AskoziaPBX webinterface.
\newline Default: \texttt{admin}
\newline Example: \texttt{-{}-askozia-user=admin}

\item {\texttt{-{}-askozia-pw=<string>}} \newline
Password for the administrator user of the AskoziaPBX.
\newline Default: \texttt{askozia}
\newline Example: \texttt{-{}-askozia-pw=askozia}

\item {\texttt{-{}-testname=<string>}} \newline
This parameter helps to keep your results directory uncluddered. All files of the
current script call (all tests, debug files etc.) are saved in the subdir \newline
\texttt{./results/<testname>/}. If undefined, the files will be saved in the
results directory directly, so it will be messy soon.
\newline Default: undefined (direct saving of results in subdir \texttt{./results})
\newline E.g. \texttt{-{}-testname=2010-01-01\_1030}
\newline (saving of results in subdir \texttt{./results/2010-01-01\_1030/})

\item {\texttt{-{}-debug}} \newline
Activates debug messages. Activates automatic saving of debug messages
in file \texttt{./results/<testname>\_<timestamp>/debug.log}, too.
Testname is specified by using the \texttt{-{}-testname} parameter.
\newline Default: undefined (no debug output)
\newline Example: \texttt{-{}-debug}

\item {\texttt{-{}-save-sipp-log}} \newline
The output generated by the testprogram \texttt{sipp} can be saved in a file for debugging.
The path to the file where the output is saved is
\texttt{./results/<testname>\_<timestamp>/sipp.log}.
Testname is specified by using the \texttt{-{}-testname} parameter.
\newline Default: undefined (output ignored)
\newline Example: \texttt{-{}-save-sipp-log}

\item {\texttt{-{}-help}} \newline
Displays a short help for using the testscript and exits immediatly.
\newline Default: undefined (no help shown)
\newline Example: \texttt{-{}-help}

\end{description}
%%%%%%%%%%%%%%%%%%%%%%%%%%%
\subsection{Dependencies}%%
%%%%%%%%%%%%%%%%%%%%%%%%%%%
This script was developed under Linux Mint 8 Helena (\url{http://www.linuxmint.com}).
It is not possible to execute this script on Windows because there were many Linux specific system commands
(like \texttt{kill}, \texttt{killall}, \texttt{which}, \texttt{date}, \texttt{id} and \texttt{ping}) used.

The script has the following dependencies:
\begin{itemize}
	\item Perl v5.10.0 (\url{http://www.perl.org})
	\item gnuplot 4.2 patchlevel 5 (\url{http://www.gnuplot.info})
\end{itemize}
